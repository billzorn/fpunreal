\documentclass[letterpaper,10pt]{article}
\usepackage[utf8]{inputenc}
\usepackage{amsmath}
\usepackage{tikz}

\title{Sinking Point: tracking floating-point precision with finitely many bits}
\author{Bill Zorn, Dan Grossman, Zach Tatlock}

\begin{document}

\maketitle

\begin{abstract}
 Some Abstract
\end{abstract}


\section{Introduction}

Floating point is hard because error is not compositional. Example. We solve this.

\section{Motivating Example - the quadratic formula}

Note that by definition, each individual operation has an error of at most 1/2 ulp. Obviously, this is not compositional, as the result has much less precision than the inputs.

Though, that is not necessarily apparent based on the output from IEEE floating point.

With sinking point, inaccurate results are also imprecise, while accurate results maintain their precision.

\section{Related work}

There are lots of projects to analyze the precision of floating point computations, track it dynamically through a computation, or remove error automatically by changing expressions. Sinking Point is a dynamic analysis - it's different because it avoids computing an expensive reference answer to determine the error.

\section{Sinking point design - rounding with w and p}

Define IEEE 754 style rounding in terms of w and p.

simple rules for + - * / sqrt (note that negation preserves precision)

\section{Results I}

Per-operation lost bit tables

\section{Results II}

Lost precision for more complex expressions, with reasonable argument ranges.

\section{Future work}



\bibliographystyle{ieeetr}
\bibliography{ref}

\end{document}
